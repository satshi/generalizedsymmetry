\documentclass[generalized_symmetry.tex]{subfiles}
\setcounter{chapter}{3}
\begin{document}

\chapter{高次形式対称性}
これまでは主に2次元の一般化対称性について見てきましたが、ここから徐々に高次元の一般化対称性について見ていくことにします。まずは、高次元での高次形式対称性と呼ばれるクラスの一般化対称性について見ていきます。

\section{高次形式対称性の定義}


\section{格子ゲージ理論の中心対称性}
\section{背景ゲージ場その1}
\section{背景ゲージ場その2:単体コホモロジー}
\section{自発的対称性の破れ}


\end{document}