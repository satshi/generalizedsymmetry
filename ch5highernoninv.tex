\documentclass[generalized_symmetry.tex]{subfiles}
\setcounter{chapter}{5}
\begin{document}

\chapter{高次元の非可逆対称性}
高次元の非可逆対称性は比較的新しい話題で、現在も盛んに研究が進んでいる分野です。高次元の非可逆対称性の系統的な調べ方はまだありません。いくつかの例が知られているだけです。高次元の非可逆対称性の発見の仕方にはいくつかあるのですが、ここでは2つの方法を紹介します。
\begin{itemize}
  \item 高次ゲージ化(higher gauging)
  \item 半空間ゲージ化(half-space gauging)
\end{itemize}
後者は\ref{sec:KWdefect}でKramers-Wannier双対性の欠陥を作るのに用いた方法の高次元版です。名前を見て分かるとおり、両方とも「ゲージ化」の方法ですので、まずは高次形式対称性のゲージ化について説明します。

\section{高次形式対称性のゲージ化}
\section{4次元\texorpdfstring{\Ztwo}{Z2}格子ゲージ理論}
\section{Maxwell理論}



\end{document}