\RequirePackage{plautopatch}
\documentclass[report,paper=a4, fontsize=12pt, line_length=16cm, number_of_lines=33,dvipdfmx]{jlreq}
\usepackage[T1]{fontenc}
\usepackage{amsmath,amssymb}
%\usepackage[bold,uplatex]{otf}
\usepackage{jlreq-deluxe}
%% Fonts
\usepackage{tgtermes,tgheros,tgcursor}
\renewcommand{\bfdefault}{bx}
\usepackage[libertine]{newtxmath}
\usepackage{hyperref}
\hypersetup{colorlinks=true,linkcolor=blue,citecolor=blue}

\usepackage{graphicx}
\graphicspath{{fig/}}

\usepackage{physics}
\usepackage{color}

\usepackage{tcolorbox}
\tcbuselibrary{breakable, skins, theorems}
%\usepackage{cleveref}

% font warningを出さないため
% \DeclareFontShape{JY2}{hgt}{b}{n}{<->ssub*hgt/bx/n}{}
% \DeclareFontShape{JY2}{hgt}{m}{it}{<->ssub*hgt/m/n}{}
% \DeclareFontShape{JT2}{hgt}{b}{n}{<->ssub*hgt/bx/n}{}
% \DeclareFontShape{JT2}{hgt}{m}{it}{<->ssub*hgt/m/n}{}

\newenvironment{myquote}{\begin{tcolorbox}[
  colback = blue!5, after = \noindent] }{\end{tcolorbox}}
\newenvironment{important}{\begin{tcolorbox}[
  colback = white,
  colframe = red!35,
  boxrule = 2mm,
  fonttitle = \bfseries,
  after = \noindent] }{\end{tcolorbox}}
\newenvironment{mycite}{\\ \qquad \textbullet\ }{\\}

\numberwithin{equation}{chapter}
%%%%%%%%%%%%%%%%%%%%%%%%%%%%%%%%%%%%%%%%%%%%%%%%%%%%%%%%%%%%%%%%%%%%%%
%                          often used macro
\newcommand{\del}{\partial}
\newcommand{\Cb}{\mathbb{C}}
\newcommand{\Zb}{\mathbb{Z}}
\newcommand{\CP}{\Cb \mathrm{P}}
\newcommand{\strong}[1]{{\sffamily \gtfamily \bfseries #1}}
\newcommand{\Ztwo}{\mbox{$\mathbb{Z}_{2}$}}

\title{一般化対称性について}
\author{山口 哲}
\date{\today}
\begin{document}
\maketitle
\tableofcontents

\section*{まえがき}

\chapter{導入}
\section{対称性とは?}
\section{対称性の使い方の例}
\section{いくつかの注意}

レビュー\cite{Shao:2023gho}

\chapter{対称性とトポロジカル欠陥}
\section{欠陥}
\section{対称性とトポロジカル欠陥その1}
\section{対称性とトポロジカル欠陥その2}
\section{対称性欠陥のまとめ}
\section{一般化対称性}



\chapter{2次元Ising模型}

\section{2次元Ising模型と\texorpdfstring{\Ztwo}{Z2}ゲージ化}
\section{双対対称性}
\section{Kramers-Wannier 双対性}
\section{Ising模型の演算子}
\section{KW欠陥}
\section{対称性の構造}
\section{応用}


\chapter{高次形式対称性}
\section{高次形式対称性の定義}
\section{格子ゲージ理論の中心対称性}
\section{背景ゲージ場その1}
\section{背景ゲージ場その2:単体コホモロジー}
\section{自発的対称性の破れ}


\chapter{高次元の非可逆対称性}
\section{高次形式対称性のゲージ化}
\section{4次元\texorpdfstring{\Ztwo}{Z2}格子ゲージ理論}
\section{Maxwell理論}

\bibliographystyle{utphys}
\bibliography{ref}
\end{document}
