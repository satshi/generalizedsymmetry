\RequirePackage{plautopatch}
\documentclass[report,paper=a4, fontsize=12pt, line_length=16cm, number_of_lines=33,dvipdfmx]{jlreq}
\usepackage[T1]{fontenc}
\usepackage{amsmath,amssymb}
%\usepackage[bold,uplatex]{otf}
\usepackage{jlreq-deluxe}
%% Fonts
\usepackage{tgtermes,tgheros,tgcursor}
\renewcommand{\bfdefault}{bx}
\usepackage[libertine]{newtxmath}
\usepackage{hyperref}
\hypersetup{colorlinks=true,linkcolor=blue,citecolor=blue}

\usepackage{graphicx}
\graphicspath{{fig/}}

\usepackage{physics}
\usepackage{color}

\usepackage{tcolorbox}
\tcbuselibrary{breakable, skins, theorems}
%\usepackage{cleveref}

% font warningを出さないため
% \DeclareFontShape{JY2}{hgt}{b}{n}{<->ssub*hgt/bx/n}{}
% \DeclareFontShape{JY2}{hgt}{m}{it}{<->ssub*hgt/m/n}{}
% \DeclareFontShape{JT2}{hgt}{b}{n}{<->ssub*hgt/bx/n}{}
% \DeclareFontShape{JT2}{hgt}{m}{it}{<->ssub*hgt/m/n}{}

\newenvironment{myquote}{\begin{tcolorbox}[
  colback = blue!5, after = \noindent] }{\end{tcolorbox}}
\newenvironment{important}{\begin{tcolorbox}[
  colback = white,
  colframe = red!35,
  boxrule = 2mm,
  fonttitle = \bfseries,
  after = \noindent] }{\end{tcolorbox}}
\newenvironment{mycite}{\\ \qquad \textbullet\ }{\\}

\newtcolorbox{emphasize}[1][]{
  colback=orange!7,
  colframe=orange!80!red,
  coltitle=white,
  title={#1},
  fonttitle=\sffamily \bfseries, 
  sharp corners,
  boxrule=0.5mm
}

\numberwithin{equation}{chapter}
%%%%%%%%%%%%%%%%%%%%%%%%%%%%%%%%%%%%%%%%%%%%%%%%%%%%%%%%%%%%%%%%%%%%%%
%                          often used macro
\newcommand{\del}{\partial}
\newcommand{\Cb}{\mathbb{C}}
\newcommand{\Zb}{\mathbb{Z}}
\newcommand{\CP}{\Cb \mathrm{P}}
\newcommand{\strong}[1]{{\sffamily \gtfamily \bfseries #1}}
\newcommand{\Ztwo}{\mbox{$\mathbb{Z}_{2}$}}
\newcommand{\Hh}{\widehat{H}}
\newcommand{\Uh}{\widehat{U}}


\title{一般化対称性について}
\author{山口 哲}
\date{\today}
\begin{document}
\maketitle
\tableofcontents

\section*{まえがき}

このノートは2023年10月に東京大学駒場で行った集中講義のノートです。集中講義の機会をくださり、有益な議論をしてくださった東京大学駒場素粒子論研究室の皆様、集中講義の参加者の皆様に感謝いたします。

\chapter{導入}
\section{対称性とは?}

この講義では、対称性の一般化について取り扱います。一般化に行くまえに普通の対称性について質問したいと思います。(場の理論において)対称性とは何でしょうか?考えてみてください。この質問は哲学的に聞こえるかもしれませんが、もっと具体的なもので、例えばみなさんが授業や教科書でどう習ったか、あるいは授業どう教えているかというものです。

答えはいろいろあると思います。ここでは、その中の3つをとりあえず書いてみます。この中に皆さんの考えた答えはあるでしょうか?
\begin{itemize}
  \item[①] 場を$\phi$、作用を$S(\phi)$とします。対称性とは変換$\phi\to\phi'$であって$S(\phi')=S(\phi)$となるものです。
  \item[②] Hamiltonianを$\Hh$とします。対称性とはUnitary演算子$\Uh$であって、Hamiltonianと交換する、つまり$\Hh \Uh =\Uh \Hh$となるものです。
  \item[③] Codim 1のトポロジカル欠陥で群構造を持つもの。  
\end{itemize}
①は、一番人気がある答えで、多くの人がこれを思い浮かべたと思います。②も量子力学では良く出てくる説明で、これを思い浮かべなかった人も言われてみれば納得してもらえると思います。③は異質ですね。これは普通の教科書には載っていないです。意味が分からなかったとしてもひとまず気にしないでください。③はこの後この講義で時間をかけて説明したいことの一つです。これが①や②と同じレベルで納得してもらえれば、この講義の目的の目的の半分くらいは達成したと言えます。

さて、①や②のような分かりやすい言い方があるのに、なぜ③のような難しい言い方を知らないといけないのでしょうか。実は①や②には次のように「大域的である」という共通の不満があるのです。
\begin{itemize}
  \item[①] 変換は時空全体で一斉に変換してみることが必要になります。宇宙を考えているなら、宇宙の始まりから未来まで、ほぼ真空のところや星の中から宇宙の果てまで一斉に変換して作用が不変か?という問でしか対称性の存在を記述できていません。
  \item[②] こちらは時間方向は考えなくても良いですが、$\Hh$も$\Uh$も空間全体に広がる巨大な演算子です。しかも系のサイズによって(Hilbert空間の次元も含めて)変わります。 両方とも局所的なものの集まりなので、局所的な性質を見れば全体をいっぺんに考えなくても良いはずなのですが、②の言い方では全体でしか考えられていません。
\end{itemize}
つまり、対称性の記述としては
\begin{emphasize}
  (大域的対称性であっても)局所的な記述が望ましい。
\end{emphasize}
ということになります。

この不満は最近になって降って湧いたわけではなく、昔からありました。そしてその解決もされています。それが、最初の質問に対する4番目の答えです。
\begin{itemize}
  \item[④] 対称性とは、(連続対称性の無限小変換の場合には)カレント$J^{\mu}(x)$であって、$\del_{\mu}J^{\mu}(x)=0$を満たすことである。
\end{itemize}
最初の問でこれを思い浮かべられた方もおられたかも知れません。また、言われてみれば皆さん納得されると思います。この記述は完全に局所的で理想通りです。場の理論で対称性を用いて様々な性質を導くときに、この記述は大変便利で、欠くことができないものです。

④の答えは非常に良いものでしたが、難点は連続対称性の無限小変換に限られることです。例えば離散的対称性の場合にはカレントは存在しませんから、④の記述はありません。実は離散的な対称性の場合にも適用できる局所的な記述が③です。
\begin{emphasize}
  ③のトポロジカル欠陥は対称性の局所的な記述を与える。  
\end{emphasize}
このような理由で③の見方は、普通の対称性に対しても有用です。

\section{対称性の使い方の例}


\section{いくつかの注意}

レビュー\cite{Shao:2023gho}

\chapter{対称性とトポロジカル欠陥}
\section{欠陥}
\section{対称性とトポロジカル欠陥その1}
\section{対称性とトポロジカル欠陥その2}
\section{対称性欠陥のまとめ}
\section{一般化対称性}



\chapter{2次元Ising模型}

\section{2次元Ising模型と\texorpdfstring{\Ztwo}{Z2}ゲージ化}
\section{双対対称性}
\section{Kramers-Wannier 双対性}
\section{Ising模型の演算子}
\section{KW欠陥}
\section{対称性の構造}
\section{応用}


\chapter{高次形式対称性}
\section{高次形式対称性の定義}
\section{格子ゲージ理論の中心対称性}
\section{背景ゲージ場その1}
\section{背景ゲージ場その2:単体コホモロジー}
\section{自発的対称性の破れ}


\chapter{高次元の非可逆対称性}
\section{高次形式対称性のゲージ化}
\section{4次元\texorpdfstring{\Ztwo}{Z2}格子ゲージ理論}
\section{Maxwell理論}

\bibliographystyle{utphys}
\bibliography{ref}
\end{document}
